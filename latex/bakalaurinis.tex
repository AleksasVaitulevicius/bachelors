\documentclass{VUMIFInfBakalaurinis}
\usepackage{algorithmicx}
\usepackage{algorithm}
\usepackage{algpseudocode}
\usepackage{amsfonts}
\usepackage{amsmath}
\usepackage{bm}
\usepackage{caption}
\usepackage{color}
\usepackage{float}
\usepackage{graphicx}
% \usepackage{hyperref}  % Nuorodų aktyvavimas
\usepackage{listings}
\usepackage{subfig}
\usepackage{url}
\usepackage{wrapfig}
\usepackage{amssymb}


% Titulinio aprašas
\university{Vilniaus universitetas}
\faculty{Matematikos ir informatikos fakultetas}
\department{Informatikos katedra}
\papertype{Baigiamasis bakalauro darbas}
\title{Algoritmas maksimaliam srautui dinaminiuose tinkluose rasti}
\titleineng{Algorithm for the maximal flow in dynamic networks}
\status{4 kurso 2 grupės studentas}
\author{Aleksas Vaitulevičius}
\supervisor{lekt. Irmantas Radavičius}
\reviewer{}
\date{Vilnius \\ \the\year}

% Nustatymai
% \setmainfont{Palemonas}   % Pakeisti teksto šriftą į Palemonas (turi būti įdiegtas sistemoje)
\bibliography{bibliografija.bib} 

\begin{document}
\maketitle

\tableofcontents

\sectionnonum{Sąvokų apibrėžimai}
\begin{enumerate}
	\item sąvoka - paaiškinimas
\end{enumerate}

\sectionnonum{Įvadas}
Paskutiniuose dviejuose dešimtmečiuose yra plačiai domimasi dinaminiais grafais. Ši sritis suteikia daug teorinių žinių, kurios gali būti pritaikytos optimizuojant tokias sritis kaip: komunikacijos tinklus, VLSI kūrimą, kompiuterinę grafiką  \cite{DynamicGraphs}.

Dinaminis grafas - tai grafas, kuriam yra galima atlikti bent vieną iš šių operacijų:
\begin{itemize}
	\item Pridėjimo
	\begin{itemize}
		\item Pridėti briauną.
		\item Pridėti viršūnę.
	\end{itemize}
	\item Atėmimo
	\begin{itemize}
		\item Atimti briauną.
		\item Atimti viršūnę.
	\end{itemize}
	\item  Papildomos operacijos priklausomai nuo grafo savybių(pavyzdžiui jei grafas yra svorinis, tai jis galėtų turėti papildomą operaciją keisti svorį).
\end{itemize}
Pagal leidžiamas operacijas dinaminiai grafai yra skirstomi į:
\begin{itemize}
	\item dalinai dinaminius grafus, kurie yra skirstomi į:
	\begin{itemize}
		\item inkrementalus - vykdoma tik pridėjimo operacija
		\item dekrementalus - Vykdoma tik atėmimas
	\end{itemize}
	\item  pilnai dinaminius grafus, kuriuose vykdomos visos operacijos.
\end{itemize}
Šiame darbe nagrinėjami pilnai dinaminiai grafai.

Visa statinio grafo problemų aibė yra dinaminio grafo problemų poaibis. Tačiau dinaminiame grafo problemos sprendimas gali būti optimizuotas, nes yra daugiau informacijos apie grafą nei statiniame grafe (pavyzdžiui, jei buvo apskaičiuotas grafo maksimalus srautas ir prie grafo buvo pridėta briauna, tai bus žinomas grafo poaibio, kuriam nepriklauso naujai pridėta briauna, maksimalus srautas). Šiame darbe nagrinėjama maksimalaus srauto problema, kuri priklauso šiai aibei.

Maksimalus srautas - tai didžiausias galimas srautas tinkle iš viršūnių $s_i$ (šaltinių) iki viršūnių $t_i$ (tikslų). Tinklas - tai orientuotas grafas $G= {V, E, u}$, kur V yra viršūnių aibė, E - briaunų aibė, o u - briaunų pralaidumų aibė$ ( u : E \rightarrow R )$. Pagal šaltinių ir tikslų skaičių ši problema yra skirstoma į:
\begin{itemize}
	\item Vienšaltinę daugtikslinę - tai srautas, kuriame yra vienas šaltinis ir daugiau nei vienas tikslas.
	\item Daugiašaltinę vientikslinę - tai srautas, kuriame yra vienas tikslas ir daugiau nei vienas šaltinis.
	\item Daugiašaltinę daugtikslinę - tai srautas, kuriame yra daugiau nei vienas šaltinis ir daugiau nei vienas tikslas.
	\item Vienšaltinę vientikslinę - tai srautas, kuriame yra vienas šaltinis ir vienas tikslas.
\end{itemize}

Šiame darbe nagrinėjama tik vienšaltinė vientikslinė problema. Tačiau tarpinėms reikšmėms gauti yra naudojama ir likusių problemų sprendimo būdai. Metodas

\section{Algoritmas maksimaliam srautui dinaminiuose tinkluose rasti}
\subsection{Grupės apibrėžimas}
Grupė - tai grupuojamo tinklo T subgrafas $G= \{V, E, u\}$. Subgrafo G šaltiniai $s_i$ yra:
\begin{enumerate}
	\item  tinklo šaltinis, jei jis yra subgrafo G viršūnių aibėje,
	\item  menamos viršūnės. Jei $\exists x: x \in V$ ir grupė $G_i : G_i \neq G$ turi viršūnę y, kuri nepriklauso grafui G, bei egzistuoja briauna $y \rightarrow x$, tai egzistuoja menama viršūnė x' ir briauna $x' \rightarrow x$, kurios talpa yra lygi grupės $G_i$ maksimaliam srautui su tikslu x.
\end{enumerate}
subgrafo G tikslai $t_i$ yra:
\begin{enumerate}
	\item  tinklo tikslas, jei jis yra subgrafo G viršūnių aibėje,
	\item  menama viršūnė. Jei $\exists x: x \in V$ ir grupė $G_i : G_i \neq G$ turi viršūnę y, kuri nepriklauso grafui G, bei egzistuoja briauna $x \rightarrow y$, tai egzistuoja menama viršūnė x' ir briauna $x \rightarrow x'$, kurios talpa yra lygi briaunos $x \rightarrow y$ talpai.
\end{enumerate}

Kiekviena tinklo T viršūnė v priklauso tik vienai grupei. Kiekviena tinklo T briauna e priklauso tik vienai grupei, nebent $e = x \rightarrow y : x \in G_i, y \in G_j, i \neq j$. Taip pat jei tinkle egzistuoja subgrafas, kuriame yra Eulerio ciklas, tai visos viršūnės priklausančios tam subgrafui turi būti vienoje grupėje.

Pavyzdys: tarkime turime tinklą $G = {V={s, a, b, c, d, t}, E={s  \rightarrow a, a \rightarrow b, b \rightarrow c. c \rightarrow d, d \rightarrow t}}$, kuris yra sugrupuotas į grupes, kurių V yra lygūs {s, a}, {b, c}, {d, t}. Šis grupavimas pavaizduotas paveikslėlyje - \ref{fig:grupavimas}.
\begin{figure}[h]
	\caption{Grupavimo pavyzdys}
	\centering
	\includegraphics[width=\textwidth]{img/grupes_pavyzdziui.png}
	\label{fig:grupavimas}
\end{figure}

Tada subgrafo {b, c} šaltinis yra menama viršūnė $s_a$, kuri yra sujungta briauna, kurios talpa yra subgrafo {s, a} maksimalus srautas iki tikslo b, o tikslas yra d.

\subsection{Fordo Fulkersono algoritmas}
Maksimaliems srautams grupėse rasti algoritmas naudoja modifikuotą Fordo Fulkersono algoritmą, kuris naudojasi BFS \cite{BFS} galimam srautui rasti. Originalus Fordo Fulkersono algoritmas \cite{FiN} yra skirtas rasti vieno šaltinio ir vieno tikslo maksimalų srautą, tačiau sukurto algoritmo atveju gali susidaryti grupės, kurios turi kelis tikslus ir arba kelis šaltinius. Tad modifikuotas Fordo Fulkersono algoritmas yra skirtas rasti kelių šaltinių ir kelių tikslų maksimalų srautą.

BFS algoritmas, aprašytas publikacijoje :
\begin{enumerate}
	\item Inicializuojami masyvai V, Q ir FLOW, į V ir Q patalpinamas tinklo šaltinis.
	\item Jei Q yra tuščias, tai einama į žingsnį 9
	\item Išimamas paskutinis masyvo Q elementas y.
	\item Jei $\nexists$ viršūnė $x : y \rightarrow x, x \notin V$, tai einama į žingsnį 2. 
	\item Briauna $y  \rightarrow x$ patalpinama į masyvą FLOW. 
	\item Viršūnė x patalpinama į masyvą V
	\item Viršūnė x patalpinama į Q masyvo pradžią. 
	\item Einama į žingsnį 4. 
	\item Baigiamas algoritmas.
\end{enumerate}

Klasikinis Fordo Fulkersono algoritmas, kuris yra aprašytas publikacijoje bei naudojantis BFS:
\begin{enumerate}
	\item Maksimaliam srautui priskiriama reikšmė nulis.Sukuriama tinklo kopija G ir inicializuojama tinklo MAX reikšmė {V=\{\}, E=\{\},u=\{\}}.
	\item Naudojant BFS randamas srautas nuo šaltinio iki tikslo.
	\item Jei nė vieno srauto nėra randama einama į žingsnį 8.
	\item Sumažinama visų briaunų, kurie priklauso rastam srautui, talpas per rasto srauto dydį tinkle G.
	\item Rastas srautas pridedamas prie tinklo MAX.
	\item Maksimalaus srauto reikšmė yra padidinama per rasto srauto dydį.
	\item Einama į žingsnį 2.
	\item Baigiamas algoritmas.
\end{enumerate}

Modifikuotas Fordo Fulkersono algoritmas, naudojantis BFS:
\begin{enumerate}
	\item Masyvo maksimalaus srauto dydžiai, kurio dydis yra lygus tikslų skaičiui, reikšmės nustatomos į nulį.Sukuriama tinklo kopija G ir inicializuojama tinklo MAX reikšmė {V=\{\}, E=\{\},u=\{\}}.
	\item Naudojant BFS randami srautai nuo visų šaltinių iki visų tikslų.
	\item Jei nė vieno srauto nėra randama einama į žingsnį 8.
	\item Sumažinama visų briaunų, kurie priklauso rastiems srautams, talpas per rasto srauto dydį tinkle G.
	\item Rasti srautai pridedamas prie tinklo MAX.
	\item Masyvo maksimalaus srauto dydžiai elementų, kurie atitinka pasiektus tikslus, reikšmės padidinamos per rastų srautų dydžius.
	\item Einama į žingsnį 2.
	\item Baigiamas algoritmas.
\end{enumerate}

\subsection{Grupavimo funkcija}
Šiame darbe tiriamas algoritmas yra paremtas Frederiksono suformuluotu grupavimo metodu \cite{DSfUoMST}. Grupavimo metodas - tai metodas, kuris yra pagrįstas grafo dalinimu į subgrafus vadinamus grupėmis. Grafas yra padalinamas taip, kad kiekviena atlikta operacija turėtų įtakos tik daliai grupių, bet ne visoms. Todėl tiriamas algoritmas veikia pagal grupavimo metodą tik tada kai yra patenkinta sąlyga: jei tinkle egzistuoja subgrafas, kuriame yra Eulerio ciklas, tai visos viršūnės priklausančios tam subgrafui yra vienoje grupėje. Šitai sąlygai patenkinti yra naudojama grupavimo funkcija, kuri naudoja šias pagalbines funkcijas:

Sąlygos tenkinimo funkcija - tinklo EG, kurio viršūnės yra grupių, tenkinančių pateiktą sąlygą, viršūnių masyvai, kūrimo funkcija.
\begin{enumerate}
	\item Iš apskaičiuojamo tinklo $C=\{V_C, E_C, u_C\}$ sukuriamas tinklas EG, kurio viršūnės būtų apskaičiuojamo tinklo viršūnės patalpintos masyvuose, o briaunos atitiktų apskaičiuojamo tinklo briaunas.
	\item Sukuriamas masyvas B, kuriame talpinamos briaunos, su kuriomis reikia daryti skaičiavimus, stekas PATH, kuriame talpinamos aplankytos viršūnės, ir viršūnių iteratorius x, jam suteikiama C šaltinio reikšmė.
	\item Jei PATH yra tuščias, einamana į žingsnį .
	\item Jei masyve B $\exists x \rightArrow y : y \in V_C$, tai sukuriamas masyvas B', į kurį yra sudedamos visos viršūnės y iš B masyvo.
	\item Jei masyve B $\nexists x \rightArrow y : y \in V_C$, tai sukuriamas masyvas B', į kurį yra sudedamos visos viršūnės y iš tinklo V.
	\item Jei B' yra tuščias tai einama į žingsnį .
	\item Iš masyvo B' yra išimamas pirmas elementas y.
	\item Jei $\nexists y \in V_C$, tai surandama viršūnė z, kuri turi visus y elementus (toliau y := z).
	\item Jei PATH neturi elemento y, tai elementas y įdedamas į PATH ir einamana į žingsnį .
	\item Inicializuojama nauja viršūnė n su visais y elementais.
	\item Iš PATH išimamas elementas z.
	\item Jei elementas z = y, tai enama į žingsnį 15 ir y = n.
	\item  Tinklo EG viršūnės z ir n yra pakeičiamos x ir n konkatenacija (toliau n yra z ir n konkatenacija).
	\item  Einama į žingsnį 11.
	\item  Visi masyvo B' elementai įdedami į masyvą B ir viršūnių iteratoriui x suteikiama y reikšmė.
	\item  Einama į žingsnį .
	\item  Iš steko PATH išimamas elementas, iteratoriui x suteikiama PATH viršutinio elemento reikšmė.
	\item  Einama į žingsnį .
\end{enumerate}

Pati grupavimo funkcija:
\begin{enumerate}
	\item Kviečiama Sąlygos tenkinimo funkcija.
\end{enumerate}


\subsection{Sukurtas algoritmas}
Sukurtas algoritmas turi dvi fazes: inicializavimo ir skaičiavimo po kiekvieno pokyčio. Inicializavimo fazėje yra  išskaidomas pateiktas tinklas į grupes, apskaičiuojami maksimalūs srautai kiekvienoje grupėje, kuri priklauso maksimaliam pateikto tinklo srautui, ir iš rastų srautų yra gaunamas maksimalus pateikto tinklo srautas. Tada  skaičiavimo po kiekvieno pokyčio fazėje yra laukiama pokyčio. Kai jis įvyksta, yra kviečiama funkcija priklausomai nuo to koks pokytis įvyko.

\subsubsection{Inicializavimo fazė}

Inicializavimo fazė (maksimalus apskaičiuojamo tinklo srautas yra grupėje su apskaičiuojamo tinklo tikslu):
\begin{enumerate}
	\item Apskaičiuojamam tinklui $C=\{V_C, E_C, u_C\}$ yra kviečiama grupavimo funkcija, kuri grąžina grupių tinklą $R\{V_R, E_R, u_R\}$.
	\item Inicializuojamas sąrašas MF, kuriame  laikoma kokiam tikslui koks maksimalus srautas buvo apskaičiuotas, ir masyvas CAL, kuriame laikomos visos grupės, kurios jau buvo apskaičiuotos.
	\item Apskaičiuojama grupės, kurioje yra tinklo C šaltinis, maksimalus srautas, kviečiant modifikuotą Fordo Fulkersono algoritmą. Rezultatas įsimenamas sąraše MF ir pačioje grupėje.
	\item Grupė G patalpinama į masyvą CAL.
	\item Jei masyve CAL yra grupė, kurioje yra tinklo C tikslas tai einama į žingsnį 11.
	\item Jei $\nexists$ grupė $G : Y \rightarrow G, \forall Y \in CAL, G \in V_R, G \notin CAL$, tai einama į žingsnį 11.
	\item Kiekvienam grupės $G : Y \rightarrow G, \forall Y \in CAL, G \in V_R, G \notin CAL$ briaunai $m \rightarrow s$, kur m yra menama viršūnė, kuri yra grupės G šaltinis, yra suteikiama talpa iš sąrašo MF elemento, kuris atitinka tikslą s.
	 \item Grupei G apskaičiuojamas maksimalus srautas naudojant modifikuotą Fordo Fulkersono algoritmą.  Rezultatas įsimenamas sąraše MF ir pačioje grupėje.
	 \item Grupė G patalpinama į masyvą CAL.
	 \item Einama į žingsnį 5. 
	 \item Algoritmo pabaiga. 
\end{enumerate}

\subsubsection{Po kiekvieno pokyčio fazė}

Dalis funkcijų, kurios yra kviečiamos po konkretaus pakeitimo, naudoja šias pagalbines funkcijas:


\begin{enumerate}
	\item 
\end{enumerate}

Jei įvyksta viršūnės x pridėjimas, tai tada yra įvykdoma funkcija:
\begin{enumerate}
	\item Jei $\exists$ grupė G, kuri neturi šaltinių ir tikslų (iki grupės nėra kelio nuo apskaičiuojamo tinklo šaltinio), tai grupei G yra pridedama viršūnė x ir einama į žingsnį 5.
	\item Sukuriama grupė N.
	\item Grupei N  yra pridedama viršūnė x.
	\item Grupė N  yra pridedama į grupių tinklą.
	\item Algoritmo pabaiga. 
\end{enumerate}

Jei įvyksta viršūnės x atėmimas, tai tada yra įvykdoma funkcija:
\begin{enumerate}
\item Jei $\exists$ grupė G, kuri neturi šaltinių ir tikslų (iki grupės nėra kelio nuo apskaičiuojamo tinklo šaltinio), tai grupei G yra pridedama viršūnė x ir einama į žingsnį 5.
\item Sukuriama grupė N.
\item Grupei N  yra pridedama viršūnė x.
\item Grupė N  yra pridedama į grupių tinklą.
\item Algoritmo pabaiga. 
\end{enumerate}

\subsection{Sukurto algoritmo trūkumai}
Taisant sukurtą algoritmą buvo pastebėta, kad algoritmas turi trūkumą. Jei apskaičiuojamas tinklas yra sugrupuojamas į 2 grupes, iš kurių viena neveda link tikslo, tai ta grupė "išsiurbia srautą".

Pavyzdžiui, jeigu yra duotas tinklas $N=\{V=\{0, 1, 2, 3, 4\}, E=\{0 \rightarrow 1, 1 \rightarrow 2, 1 \rightarrow 4, 2 \rightarrow 3, 3 \rightarrow 2\}, u=\{(0 \rightarrow 1) = 10, (1 \rightarrow 2) = 5, (2 \rightarrow 3) = 5, (3 \rightarrow 2) = 10, (1 \rightarrow 4) = 10\}\}$, kurio šaltinis yra 0, o tikslas 4  \ref{fig:trukumas}.
\begin{figure}[h]
	\caption{Tinklo, kurio maksimalaus srauto algoritmas negali korektiškai apskaičiuoti, pavyzdys}
	\centering
	\includegraphics[width=0.5\textwidth]{img/trukumas.png}
	\label{fig:trukumas}
\end{figure}

Grupavimo funkcija sugrupuoja tinklą $N$ į grupes $G_1=\{V=\{0, 1, 4\}, E={0 \rightarrow 1, 1 \rightarrow 4, 1 \rightarrow m2}, u=\{(0 \rightarrow 1) = 10, (1 \rightarrow m2) = 5, (1 \rightarrow 4) = 10\}\}$ ir  $G_2=\{V=\{m2, 2, 3\}, E={m2 \rightarrow 2, 2 \rightarrow 3, 3 \rightarrow2}, u=\{(2 \rightarrow 3) = 5, (3 \rightarrow 2) = 10\}\}$, kur grupės $G_1$ šaltinis yra 0,  tikslai - m2 ir 4, o grupės $G_2$ šaltinis yra m2, tikslo grupė $G_2$ neturi  \ref{fig:trukumoGrupes}.
\begin{figure}[h]
	\caption{Tinklo, kurio maksimalaus srauto algoritmas negali korektiškai apskaičiuoti, pavyzdys}
	\centering
	\includegraphics[width=0.5\textwidth]{img/trukumoGrupes.png}
	\label{fig:trukumoGrupes}
\end{figure}

Tada apskaičiuojant grupės $G_1$ maksimalius srautus bus gauti rezultatai: maksimalus srautas su tikslu m2 yra lygus 5, o su tikslu 4 - 5. Grupė $G_2$ neturi tikslo, tad su ja jokių skaičiavimų nevykdoma. Tad apskaičiuotas tinklo srautas yra 5, kai tikras tinklo N srautas yra 10. Tad algoritmo korektiškam veikimui užtikrinti reikia nustatyti ar grupavimo funkcija sugrupuoja tinklą į grupes, su kuriomis algoritmas galėtų rasti korektišką maksimalų srautą.

\section{Sukurto algoritmo įgyvendinimas}
Tam, kad pasiekti šio darbo tikslą buvo įgyvendintas sukurtas algoritmas. Įgyvendinimas buvo parašytas JAVA kalba versija 9, visos naudojamos bibliotekos yra įdiegiamos į algoritmo įgyvendinimą naudojantis įrankiu MAVEN versija 3. Šis įgyvendinimas naudojasi šių bibliotekų funkcionalumu: 
\begin{enumerate}
	\item Jgrapht - ši biblioteka yra naudojama grafų saugojimui, generavimui ir skaičiavimams.
	\item Jgraphx - ši biblioteka yra naudojama grafų atspausdinimui vartotojo grafinėje sąsajoje.
	\item lombok - naudojama kodo generavimui, kuris yra skirtas pasiekti ir modifikuoti privačius laukus.
	\item junit - naudojamas kodo modulių testavimui.
\end{enumerate}

Įgyvendinto algoritmo architektūra pavaizduota \ref{fig:architecture}. Šioje architektūroje kiekviena klasė atlieka šias funkcijas:
\begin{enumerate}
	\item DynamicNetworkWithMaxFlowAlgorithm - šioje klasėje yra laikomas apskaičiuojamas dinaminis algoritmas NETWORK, yra atliekami pokyčiai tinklui NETWORK ir yra atliekama pagrindinės sukurto algoritmo funkcijos.
	\item FordFulkerson - ši klasė atlieka modifikuoto Fordo Fulkersono algoritmo funkciją.
	\item BFS - ši klasė atlieka paieška platyn algoritmo funkciją.
	\item DividerToClusters - ši klasė atlieka grupavimo funkciją.
	\item Network - ši klasė yra tinklas.
	\item DynamicNetwork - ši klasė yra dinaminis tinklas.
	\item EulerCycleWarps - ši klasė yra tinklas, kurio viršūnės yra grupių, tenkinančių sąlygą subgrafai su Eulerio ciklais, viršūnių masyvai.
	\item ClustersNetwork - ši klasė yra grupių tinklas.
\end{enumerate}
Klasės SimpleDirectedGraph<List<int>, DefaultEdge> ir SimpleDirectedWeightedGraph<int, WeightedEdge> yra klasės iš jgrapht bibliotekos. Įgyvendinimo architektūra yra pavaizduota prieduose poskyryje Įgyvendinto algoritmo architektūra \ref{fig:architecture}, \ref{fig:architecture0}, \ref{fig:architecture1} ir  \ref{fig:architecture2}.

\section{Empiriniai bandymai}
Tam, kad nustatyti ar verta formaliai įrodyti sukurtą algoritmą buvo atlikti empiriniai bandymai, kurie skirti ištirti algoritmo korektišką veikimą ir efektyvumą. Šiems bandymams atlikti buvo įgyvendintas sukurtas algoritmas, modulis, generuojanti tinklus, ir modulis, kuris atlieka bandymus su tinklus generuojančio modulio rezultatais. Su bandymus atliekančio modulio rezultatais yra atliekami statistiniai skaičiavimai.

\subsection{Tinklus generuojantis modulis}

Šiame darbe reikia ištirti kuo įvairesnius tinklus, iš kurių parametrų būtų paprasta sukonstruoti regresinius modelius. Tad tinklus generuojantis modulis turi generuoti tinklus $N_i = \{V_{N_i}, E_{N_i}, u{N_i}\}$, kurių parametrai tenkintų šias sąlygas:
\begin{itemize}
	\item Viršūnių aibių  $V_{N_i}$ dydžių aibės $SV$ augimo greitis būtų linijinis ir $SV$ turi būti baigtinė.
	\item Kiekvieną kartą generuojant tinklą $N_i$ yra tikimybė sugeneruoti jungų tinklą.
	\item Sugeneruotų tinklų aibėje egzistuoja tinklai su skirtingais viršūnių aibių dydžiais ir vidutiniu galimų briaunų skaičiumi. Vidutinis galimų briaunų skaičius yra apskaičiuojamas $SE_a = \frac{SE_{max} + SE_{min}}{2}$, kur  $SE_{max}$ yra maksimalus galimų briaunų skaičius, o $SE_{min}$ - minimalus galimų briaunų skaičius.
	\item Sugeneruotų tinklų aibėje egzistuoja tinklai su skirtingais viršūnių aibių dydžiais ir vidutiniškai mažesniu briaunų skaičiumi negu vidutinis galimų briaunų skaičius.  Tinklo briaunų skaičius yra laikomas vidutiniškai mažesniu negu vidutinis galimų briaunų skaičius, jei tenkinama sąlyga: $SE_{a_{min}} = \frac{SE_a + SE_{min}}{2}$, kur $SE_a$ yra vidutinis galimų briaunų skaičius, o $SE_{min}$ - minimalus galimų briaunų skaičius.
	\item Sugeneruotų tinklų aibėje egzistuoja tinklai skirtingais viršūnių aibių dydžiais ir vidutiniškai didesniu briaunų skaičiumi negu vidutinis galimų briaunų skaičius.  Tinklo briaunų skaičius yra laikomas vidutiniškai didesniu negu vidutinis galimų briaunų skaičius, jei tenkinama sąlyga: $SE_{a_{max}} = \frac{SE_a + SE_{max}}{2}$, kur $SE_a$ yra vidutinis galimų briaunų skaičius, o $SE_{max}$ - maksimalus galimų briaunų skaičius.
\end{itemize}
Žinant, kad minimalus galimų briaunų skaičius tinkle yra $SE_{min}(SV_i)  = SV_i - 1$, o maksimalus galimų briaunų skaičius tinkle yra $SE_{max}(SV_i) = SV_i \times (SV_i - 1)$, kur $SV_i$ yra tinko viršūnių skaičius, tai gauname, kad  $SE_{a}(SV_i) =\frac{SV_i^2 - 1}{2}$, $SE_{a_{min}}(SV_i)  = \frac{SV_i^2 + 2 \times SV_i - 3}{4}$ ir $SE_{a_{max}}(SV_i) = \frac{3 \times SV_i^2 - 2 \times SV_i - 1}{4}$.

Tad tinklus generuojantis modulis sugeneruoja tinklų $N_i = \{V_{N_i}, E_{N_i}, u{N_i}\}$ aibę A. Aibėje A yra 10 poaibių $A'_j$, kurių tinklų viršūnių aibių $V_{N_i}$ dydžiai $SV_i$ yra lygūs. Tad kiekvienas poaibis  $A'_j$ turi skirtingą dydį  $SV_j$. Modulis sugeneruoja poaibius $A'_j$ su šiais $SV_j = 10 \times j : j = 1 .. 10$. Kiekviename poaibyje $A'_j$ yra po 3 poaibius $A''_j$, kuriuose yra po 10 tinklų $N_i$, kurių briaunų aibių $E_{N_i}$ dydžiai yra lygūs. Tad kiekvienas poaibis  $A''_j$ turi skirtingą dydį  $SE_j$. Modulis sugeneruoja poaibius $A''_j$ su šiais  $SE_j : SE_{a}(SV_i), SE_{a_{min}}(SV_i) , SE_{a_{max}}(SV_i)$.

\subsection{Bandymus atliekantis modulis}

Šiame darbe buvo sukurtas modulis ,kuris atlieka bandymus su kiekvienu tinklu, kurį sugeneravo tinklus generuojantis modulis. Šių bandymų metu yra skaičiuojamas briaunų panaudotų skaičiavimuose skaičius. Šio bandymo rezultatai yra masyvai INCORRECT, ACTION, $Algorithm_{\{T\}}$ ir $Test_{\{T\}}$, kur T yra kiekvieno dinaminio tinklo operacijos tipas. Šio bandymo eiga su tinklu NETWORK:

\begin{enumerate}
	\item Apskaičiuojamas tinklo NETWORK maksimalus srautas ACTUAL, naudojant sukurto algoritmo įgyvendinimą.
	\item Apskaičiuojamas tinklo NETWORK maksimalus srautas EXPECTED, naudojant modifikuotą Fordo Fulkersono algoritmą.
	\item Jei $ACTUAL \neq EXPECTED$, tai į masyvą INCORRECT įdedama grupė G, o į masyvą ACTION reikšmė \textit{INIT}.
	\item Su kiekvienu dinaminio tinklo operacijos tipu TYPE yra atliekami šie veiksmai:
	\begin{enumerate}
		\item Tinklui NETWORK atliekama operacija TYPE.
		\item Apskaičiuojamas tinklo NETWORK maksimalus srautas ACTUAL, naudojant sukurto algoritmo įgyvendinimą. Į $Algorithm_{TYPE}$ įdedamas skaičiavime panaudotų briaunų skaičius.
		\item Apskaičiuojamas tinklo NETWORK maksimalus srautas EXPECTED, naudojant modifikuotą Fordo Fulkersono algoritmą. Į $Test_{TYPE}$ įdedamas skaičiavime panaudotų briaunų skaičius.
		\item Jei $ACTUAL \neq EXPECTED$, tai į masyvą INCORRECT įdedamas tinklas NETWORK, o į masyvą ACTION įdedamas TYPE.
	\end{enumerate}

\end{enumerate}

\subsection{Statistiniai skaičiavimai ir jų rezultatai}

Naudojantis atliktų bandymų rezultatu, INCORRECT ir ACTION masyvais, buvo nustatyta, kad iš 300 apskaičiuotų tinklų 3 tinklams $N_i$ buvo apskaičiuotas neteisingas maksimalus srautas. Visų 3 tinklų $N_i$ viršūnių aibių dydžiai yra 10. Dviejų tinklų iš tinklų $N_i$ briaunų skaičius yra vidutiniškai mažesnis už vidutinį galimų briaunų skaičių, likusio tinklo briaunų skaičius yra vidutinis galimų briaunų skaičius. Todėl taip pat buvo nekorektiškai apskaičiuoti šių tinklų maksimalūs srautai po kiekvieno pokyčio. Tačiau žinant, kad maksimalus srautas buvo nekorektiškai apskaičiuotas inicializavimo fazėje, tai galima teigti, kad yra nežinoma ar algoritmas po kiekvieno pokyčio apskaičiavo maksimalų srautą korektiškai.

Taip pat naudojantis atliktų bandymų rezultatu, INCORRECT ir ACTION masyvais, buvo nustatyta, kad iš 300 apskaičiuotų tinklų tinklo $N$ maksimalus srautas buvo apskaičiuotas nekorektiškai po talpos pakeitimo pokyčio. Tinklo $N$ viršūnių skaičius yra 60, o briaunų skaičius yra vidutiniškai mažesnis nei vidutinis galimų briaunų skaičius.

Naudojantis atliktų bandymų rezultatu, masyvais $Algorithm_{\{T\}}$ ir $Test_{\{T\}}$,  sukurtam algoritmui ir Fordo Fulkersono algoritmui buvo sukonstruoti paprasti linijiniai modeliai $USED_EDGES ~ VIRSUNES + BRIAUNOS$ su kiekvienu pokyčio tipu T, kur USED\_EDGES - tai panaudotų briaunų skaičius, VIRSUNES - apskaičiuoto tinklo viršūnių skaičius, BRIAUNOS - apskaičiuoto tinklo briaunų skaičius (gali įgyti reikšmes: VIDUTINIS - vidutinis galimų briaunų skaičius, MAZAI - vidutiniškai mažesnis negu vidutinis galimų briaunų skaičius, DAUG - vidutiniškai didesnis negu vidutinis galimų briaunų skaičius). Sukonstruoto algoritmo modeliai:
\begin{enumerate}
	\item Po viršūnės pridėjimo: $USED_EDGES = 0$
	\item Po briaunos pridėjimo: $USED_EDGES = 5905.3VIRŠŪNĖS + 16253DAUG - 9611.2MAZAI + VIDUTINIS - 162425.5$, tik parametras USED\_EDGES ir konstanta turi dideles įtakas.
	\item Po viršūnės atėmimo: $USED_EDGES = 5911.6VIRŠŪNĖS + 15370DAUG - 10513.6MAZAI + VIDUTINIS - 161921.6$, tik parametras USED\_EDGES ir konstanta turi dideles įtakas.
	\item Po briaunos atėmimo: $USED_EDGES = 5906VIRŠŪNĖS + 15575DAUG -10436MAZAI + VIDUTINIS - 161921$, tik parametras USED\_EDGES ir konstanta turi dideles įtakas.
	\item Po talpos pakeitimo: $USED_EDGES = 5912.6VIRŠŪNĖS + 16059.5DAUG - 10430.8MAZAI + VIDUTINIS - 162265.2$, tik parametras USED\_EDGES ir konstanta turi dideles įtakas.
\end{enumerate}
Fordo Fulkersono algoritmo modeliai:
\begin{enumerate}
	\item Po viršūnės pridėjimo: $USED_EDGES = 6103.5VIRŠŪNĖS + 14124.5DAUG -9955.4MAZAI + VIDUTINIS - 166184.7$, tik parametras USED\_EDGES ir konstanta turi dideles įtakas.
	\item Po briaunos pridėjimo: $USED_EDGES = 5924.9VIRŠŪNĖS + 14406.2DAUG - 8848.7MAZAI + VIDUTINIS - 162945.9$, tik parametras USED\_EDGES ir konstanta turi dideles įtakas.
	\item Po viršūnės atėmimo: $USED_EDGES = 5923.9VIRŠŪNĖS + 14405.7DAUG - 8885.6MAZAI + VIDUTINIS - 162909.8$, tik parametras USED\_EDGES ir konstanta turi dideles įtakas.
	\item Po briaunos atėmimo: $USED_EDGES = 5923.7VIRŠŪNĖS + 14359.6DAUG - 8831.3MAZAI + VIDUTINIS - 162960.8$, tik parametras USED\_EDGES ir konstanta turi dideles įtakas.
	\item Po talpos pakeitimo: $USED_EDGES =5925.3VIRŠŪNĖS + 14430.2DAUG -8774.7MAZAI + VIDUTINIS - 163061.4$, tik parametras USED\_EDGES ir konstanta turi dideles įtakas.
\end{enumerate}

\sectionnonum{Išvados ir rezultatai}
Šio darbo rezultatas yra įgyvendintas kursiniame darbe sukurtas algoritmas ir atlikti empiriniai eksperimentai su sukurto algoritmo įgyvendinimu. Papildomai šiame darbe buvo koreguojamas kursiniame darbe sukurtas algoritmas ir nustatytas sukurto algoritmo trūkumas.

Naudojantis atliktais empirinių eksperimentų rezultatais, galima daryti išvadą, kad sukurtą algoritmą formaliai įrodinėti yra neprasminga. Sukurtas algoritmas korektiškai veikia tik ieškodamas specifinių tinklų, kuriuos grupavimo funkcija sugrupuoja į grupes, iš kurių galima pasiekti tinklo tikslą, maksimalaus srauto. Taip pat sukurtas algoritmas nėra efektyvesnis už Fordo Fulkersono algoritmą, kuris yra skirtas ieškoti maksimalių srautų statiniuose tinkluose. Tad net pritaikius sukurtą algoritmą visiems dinaminiams tinklams, pakeičiant grupavimo funkciją, sukurtas algoritmas nebūtų naudingas.

\sectionnonum{Conclusions and results}
Result of this work is implementation of algorithm, developed in course work, and performed empirical experiments. Additionally, the developed algorithm was improved and algorithm's weakness was determined.

Performed empirical experiment results concludes that formal prove of correct algorithm performance is useless. As the developed algorithm finds correct maximum flow only of specific networks, which can be clustered into clusters, which has route to sink of the network, by using clustering function. Moreover, the developed algorithm is not more efficient than Ford Fulkerson algorithm, which is used to find maximum flow in static networks. In conclusion, even if the algorithm was adapted to find maximum flow in all networks by changing clustering function, the developed algorithm would still be unimportant.


\printbibliography[heading=bibintoc] % Literatūros šaltiniai aprašomi
% bibliografija.bib faile. Šaltinių sąraše nurodoma panaudota literatūra,
% kitokie šaltiniai. Abėcėlės tvarka išdėstoma tik darbe panaudotų (cituotų,
% perfrazuotų ar bent paminėtų) mokslo leidinių, kitokių publikacijų
% bibliografiniai aprašai (šiuo punktu pasirūpina LaTeX). Aprašai pateikiami
% netransliteruoti.

\appendix  % Priedai
% Prieduose gali būti pateikiama pagalbinė, ypač darbo autoriaus savarankiškai
% parengta, medžiaga. Savarankiški priedai gali būti pateikiami kompiuterio
% diskelyje ar kompaktiniame diske. Priedai taip pat vadinami ir numeruojami.
% Tekstas su priedais siejamas nuorodomis (pvz.: \ref{img:mlp}).

\section{Įgyvendinto algoritmo architektūra}
\begin{figure}[h]
	\caption{Sukurto algoritmo įgyvendinimo UML klasių diagrama}
	\centering
	\includegraphics[width=\textwidth]{img/architecture.png}
	\label{fig:architecture}
\end{figure}

\begin{figure}[h]
	\caption{Sukurto algoritmo įgyvendinimo UML klasių diagrama}
	\centering
	\includegraphics[width=1.5\textwidth, height=0.5\textheight, angle=90]{img/architecture0.png}
	\label{fig:architecture0}
\end{figure}

\begin{figure}[h]
	\caption{Sukurto algoritmo įgyvendinimo UML klasių diagrama}
	\centering
	\includegraphics[width=1.5\textwidth, height=0.6\textheight, angle=90]{img/architecture1.png}
	\label{fig:architecture1}
\end{figure}

\begin{figure}[h]
	\caption{Sukurto algoritmo įgyvendinimo UML klasių diagrama}
	\centering
	\includegraphics[width=\textwidth, height=0.6\textheight]{img/architecture2.png}
	\label{fig:architecture2}
\end{figure}

\section{Atliktų bandymų rezultatai}
% tablesgenerator.com - converts calculators (e.g. excel) tables to LaTeX
\begin{table}[H]\footnotesize
	\centering
	\caption{Eksperimento metu vidutiniškai panaudotų briaunų skaičius po viršūnės ir briaunos pridėjimų operacijų}
	\label{experiment:add}
	\begin{tabular}{l l | l l | l l}
		\multicolumn{2}{l}{}                                                                                                     & \multicolumn{2}{| l}{\begin{tabular}[c]{@{}l@{}}Vidutiniškai panaudotų briaunų skaičius\\ skaičiavimuose po viršūnės pridėjimo\end{tabular}} & \multicolumn{2}{| l}{\begin{tabular}[c]{@{}l@{}}Vidutiniškai panaudotų briaunų skaičius\\ skaičiavimuose po briaunos pridėjimo\end{tabular}} \\
		\hline
		\begin{tabular}[c]{@{}l@{}}Viršūnių\\ skaičius\end{tabular} & \begin{tabular}[c]{@{}l@{}}Briaunų\\ skaičius\end{tabular} & \begin{tabular}[c]{@{}l@{}}Sukurtas\\ algoritmas\end{tabular}    & \begin{tabular}[c]{@{}l@{}}Fordo Fulkersono\\ algoritmas\end{tabular}   & \begin{tabular}[c]{@{}l@{}}Sukurtas\\ algoritmas\end{tabular}    & \begin{tabular}[c]{@{}l@{}}Fordo Fulkersono\\ algoritmas\end{tabular}   \\
		\hline
		10                                                          & $SE_{a}(SV_i)$                                                    & 0                                                                &      403.2                                                            & 254.1                                                            & 272.6                                                                   \\
		20                                                          & $SE_{a}(SV_i)$                                                    & 0                                                                &     4204.1                                                               & 3489.9                                                           & 3579.9                                                                  \\
		30                                                          & $SE_{a}(SV_i)$                                                    & 0                                                                &   12999.9                                                                & 11427.5                                                          & 11651.6                                                                 \\
		40                                                          & $SE_{a}(SV_i)$                                                    & 0                                                                &     37038                                                            & 32989.5                                                          & 33657.8                                                                 \\
		50                                                          & $SE_{a}(SV_i)$                                                    & 0                                                                &     69779.8                                                                & 54543.4                                                          & 64847.9                                                                 \\
		60                                                          & $SE_{a}(SV_i)$                                                    & 0                                                                &    126870.2                                                              & 119681.4                                                         & 120086.6                                                                \\
		70                                                          & $SE_{a}(SV_i)$                                                    & 0                                                                &    195947.7                                                           & 183152.2                                                         & 186273.9                                                                \\
		80                                                          & $SE_{a}(SV_i)$                                                    & 0                                                                &     273230.5                                                              & 270339.2                                                         & 265640.1                                                                \\
		90                                                          & $SE_{a}(SV_i)$                                                    & 0                                                                &    387730.5                                                           & 393225.4                                                         & 377430.3                                                                \\
		100                                                         & $SE_{a}(SV_i)$                                                    & 0                                                                &      587174.1                                                             & 569387.1                                                         & 579862.7                                                                \\
		10                                                          & $SE_{a_{max}}(SV_i)$                                                        & 0                                                                &    455.6                                                            & 283.2                                                            & 280                                                                     \\
		20                                                          & $SE_{a_{max}}(SV_i)$                                                        & 0                                                                &     4196.8                                                               & 3547                                                             & 3639.6                                                                  \\
		30                                                          & $SE_{a_{max}}(SV_i)$                                                        & 0                                                                &    16336.5                                                                & 14722.3                                                          & 14619.9                                                                 \\
		40                                                          & $SE_{a_{max}}(SV_i)$                                                        & 0                                                                &   37227                                                              & 33411.7                                                          & 34012.1                                                                 \\
		50                                                          & $SE_{a_{max}}(SV_i)$                                                        & 0                                                                &    71391.5                                                                 & 68148                                                            & 66928.1                                                                 \\
		60                                                          & $SE_{a_{max}}(SV_i)$                                                        & 0                                                                &     134473.8                                                             & 127730.2                                                         & 126204.6                                                                \\
		70                                                          & $SE_{a_{max}}(SV_i)$                                                        & 0                                                                &   211475.6                                                                & 200839.2                                                         & 201210.4                                                                \\
		80                                                          & $SE_{a_{max}}(SV_i)$                                                        & 0                                                                &  329757.3                                                             & 305520.5                                                         & 311778.3                                                                \\
		90                                                          & $SE_{a_{max}}(SV_i)$                                                        & 0                                                                &    447748.1                                                            & 444358.6                                                         & 435930.5                                                                \\
		100                                                         & $SE_{a_{max}}(SV_i)$                                                        & 0                                                                &    617301.8                                                              & 599604.6                                                         & 599394.7                                                                \\
		10                                                          & $SE_{a_{min}}(SV_i)$                                                      & 0                                                                &  406.9                                                                 & 278.5                                                            & 278.9                                                                   \\
		20                                                          & $SE_{a_{min}}(SV_i)$                                                      & 0                                                                &  3825.8                                                               & 3449.6                                                           & 3277.2                                                                  \\
		30                                                          & $SE_{a_{min}}(SV_i)$                                                      & 0                                                                &  13506.5                                                                   & 11686.8                                                          & 12052.3                                                                 \\
		40                                                          & $SE_{a_{min}}(SV_i)$                                                      & 0                                                                &   30930.3                                                              & 27505.4                                                          & 28533.8                                                                 \\
		50                                                          & $SE_{a_{min}}(SV_i)$                                                      & 0                                                                &  62529.9                                                               & 56139.7                                                          & 59130.3                                                                 \\
		60                                                          & $SE_{a_{min}}(SV_i)$                                                      & 0                                                                &  120555.4                                                                 & 114535.1                                                         & 114123.6                                                                \\
		70                                                          & $SE_{a_{min}}(SV_i)$                                                      & 0                                                                &  173827.2                                                               & 169573.5                                                         & 168211.7                                                                \\
		80                                                          & $SE_{a_{min}}(SV_i)$                                                      & 0                                                                &  286708.1                                                             & 270481.1                                                         & 273976.7                                                                \\
		90                                                          & $SE_{a_{min}}(SV_i)$                                                      & 0                                                                &  382103.8                                                            & 378158.1                                                         & 371748.5                                                                \\
		100                                                         & $SE_{a_{min}}(SV_i)$                                                      & 0                                                                & 525811.8                                                              & 509642.9                                                         & 512723.8                                                               
	\end{tabular}
\end{table}

\begin{table}[H]\footnotesize
	\centering
	\caption{Eksperimento metu vidutiniškai panaudotų briaunų skaičius po viršūnės ir briaunos atėmimų operacijų}
	\label{experiment:remove}
	\begin{tabular}{l l | l l | l l}
		\multicolumn{2}{l}{}                                                                                                     & \multicolumn{2}{| l}{\begin{tabular}[c]{@{}l@{}}Vidutiniškai panaudotų briaunų skaičius\\ skaičiavimuose po viršūnės atėmimo\end{tabular}} & \multicolumn{2}{| l}{\begin{tabular}[c]{@{}l@{}}Vidutiniškai panaudotų briaunų skaičius\\ skaičiavimuose po briaunos atėmimo\end{tabular}} \\
		\hline
		\begin{tabular}[c]{@{}l@{}}Viršūnių\\ skaičius\end{tabular} & \begin{tabular}[c]{@{}l@{}}Briaunų\\ skaičius\end{tabular} & \begin{tabular}[c]{@{}l@{}}Sukurtas\\ algoritmas\end{tabular}    & \begin{tabular}[c]{@{}l@{}}Fordo Fulkersono\\ algoritmas\end{tabular}   & \begin{tabular}[c]{@{}l@{}}Sukurtas\\ algoritmas\end{tabular}    & \begin{tabular}[c]{@{}l@{}}Fordo Fulkersono\\ algoritmas\end{tabular}   \\
		\hline
		10                                                          & $SE_{a}(SV_i)$                                                    & 278.5                     & 271.8                     & 265.3                   & 265.3                   \\
		20                                                          & $SE_{a}(SV_i)$                                                    & 3544.6                    & 3572.7                    & 3533.9                  & 3561.4                  \\
		30                                                          & $SE_{a}(SV_i)$                                                    & 11415.7                   & 11639.1                   & 11402.7                 & 11605.8                 \\
		40                                                          & $SE_{a}(SV_i)$                                                    & 32971.3                   & 33638.9                   & 32951.5                 & 33547                   \\
		50                                                          & $SE_{a}(SV_i)$                                                    & 66403.2                   & 64838.4                   & 66362.9                 & 64799.1                 \\
		60                                                          & $SE_{a}(SV_i)$                                                    & 119561.9                  & 119966.8                  & 119619.8                & 120024.9                \\
		70                                                          & $SE_{a}(SV_i)$                                                    & 182429.4                  & 185897.1                  & 182738.4                & 185859.9                \\
		80                                                          & $SE_{a}(SV_i)$                                                    & 270313.2                  & 265614.4                  & 269957                  & 265568                  \\
		90                                                          & $SE_{a}(SV_i)$                                                    & 393164.6                  & 377371.2                  & 393138.6                & 377346.6                \\
		100                                                         & $SE_{a}(SV_i)$                                                    & 569117.2                  & 579592.6                  & 569283.9                & 579757.8                \\
		10                                                          &  $SE_{a_{max}}(SV_i)$                                                        & 291                       & 291                       & 251.5                   & 248.4                   \\
		20                                                          &  $SE_{a_{max}}(SV_i)$                                                        & 3548.8                    & 3651.1                    & 3522.9                  & 3569.3                  \\
		30                                                          &  $SE_{a_{max}}(SV_i)$                                                        & 14705.5                   & 14603.2                   & 14769.6                 & 14472.9                 \\
		40                                                          &  $SE_{a_{max}}(SV_i)$                                                        & 33383                     & 33981.9                   & 33374.1                 & 33974.6                 \\
		50                                                          &  $SE_{a_{max}}(SV_i)$                                                        & 68140                     & 66919.6                   & 68099.7                 & 66880.4                 \\
		60                                                          &  $SE_{a_{max}}(SV_i)$                                                        & 127704.1                  & 126178.4                  & 127667.8                & 126143.4                \\
		70                                                          &  $SE_{a_{max}}(SV_i)$                                                        & 200796.4                  & 201167                    & 200768.6                & 201139.7                \\
		80                                                          &  $SE_{a_{max}}(SV_i)$                                                        & 305478.6                  & 311735.7                  & 306650.1                & 313210.7                \\
		90                                                          &  $SE_{a_{max}}(SV_i)$                                                        & 444327.9                  & 435900.3                  & 443108.6                & 435064.7                \\
		100                                                         &  $SE_{a_{max}}(SV_i)$                                                        & 599580                    & 599370.2                  & 599501.5                & 599291.7                \\
		10                                                          & $SE_{a_{min}}(SV_i)$                                                      & 276.2                     & 276.6                     & 262.8                   & 263.2                   \\
		20                                                          & $SE_{a_{min}}(SV_i)$                                                      & 3442.5                    & 3270.8                    & 3421.7                  & 3250.3                  \\
		30                                                          & $SE_{a_{min}}(SV_i)$                                                      & 11676.4                   & 12041.9                   & 11662                   & 12027.2                 \\
		40                                                          & $SE_{a_{min}}(SV_i)$                                                      & 27281.8                   & 28453.3                   & 27513                   & 28649.7                 \\
		50                                                          & $SE_{a_{min}}(SV_i)$                                                      & 56106.2                   & 59096.1                   & 56096.1                 & 59085                   \\
		60                                                          & $SE_{a_{min}}(SV_i)$                                                      & 114491.2                  & 114080.5                  & 114475.8                & 114064.7                \\
		70                                                          & $SE_{a_{min}}(SV_i)$                                                      & 169559.8                  & 168197.6                  & 169509.1                & 168147.7                \\
		80                                                          & $SE_{a_{min}}(SV_i)$                                                      & 270464.1                  & 273959.3                  & 270100.7                & 273897.7                \\
		90                                                          & $SE_{a_{min}}(SV_i)$                                                      & 378134.5                  & 371725.1                  & 378070.7                & 371662.6                \\
		100                                                         & $SE_{a_{min}}(SV_i)$                                                      & 509592.6                  & 512673.2                  & 507157.3                & 510469                 
	\end{tabular}
\end{table}

\begin{table}[H]\footnotesize
	\centering
	\caption{Eksperimento metu vidutiniškai panaudotų briaunų skaičius po briaunos talpos pakeitimo operacijos}
	\label{experiment:update}
	\begin{tabular}{ll | ll}
		\multicolumn{2}{l}{}                 & \multicolumn{2}{| l}{Vidutiniškai panaudotų briaunų skaičius} \\
		\hline
		Viršūnių skaičius & Briaunų skaičius & Sukurtas algoritmas      & Fordo Fulkersono algoritmas      \\
		\hline
		10                & $SE_{a}(SV_i)$          & 266.8                    & 263.6                            \\
		20                & $SE_{a}(SV_i)$          & 3575.6                   & 3603.2                           \\
		30                & $SE_{a}(SV_i)$          & 11478.8                  & 11563.9                          \\
		40                & $SE_{a}(SV_i)$          & 32951.5                  & 33547                            \\
		50                & $SE_{a}(SV_i)$          & 66535.7                  & 65030.9                          \\
		60                & $SE_{a}(SV_i)$          & 119619.8                 & 120024.9                         \\
		70                & $SE_{a}(SV_i)$          & 182738.4                 & 185859.9                         \\
		80                & $SE_{a}(SV_i)$          & 269957                   & 265568                           \\
		90                & $SE_{a}(SV_i)$          & 393138.6                 & 377346.6                         \\
		100               & $SE_{a}(SV_i)$          & 569761.5                 & 580229.8                         \\
		10                & $SE_{a_{max}}(SV_i)$              & 253.4                    & 250.3                            \\
		20                & $SE_{a_{max}}(SV_i)$              & 3512.7                   & 3569.3                           \\
		30                & $SE_{a_{max}}(SV_i)$              & 14866.8                  & 14472.9                          \\
		40                & $SE_{a_{max}}(SV_i)$              & 33374.1                  & 33868.5                          \\
		50                & $SE_{a_{max}}(SV_i)$              & 68099.7                  & 66880.4                          \\
		60                & $SE_{a_{max}}(SV_i)$              & 127666                   & 126885.3                         \\
		70                & $SE_{a_{max}}(SV_i)$              & 200768.6                 & 201139.7                         \\
		80                & $SE_{a_{max}}(SV_i)$              & 307552.1                 & 313210.9                         \\
		90                & $SE_{a_{max}}(SV_i)$              & 443108.6                 & 435064.7                         \\
		100               & $SE_{a_{max}}(SV_i)$              & 601668.6                 & 600715.3                         \\
		10                & $SE_{a_{min}}(SV_i)$            & 262.8                    & 263.2                            \\
		20                & $SE_{a_{min}}(SV_i)$            & 3403.9                   & 3374.4                           \\
		30                & $SE_{a_{min}}(SV_i)$            & 11662                    & 12027.3                          \\
		40                & $SE_{a_{min}}(SV_i)$            & 27762                    & 28543.5                          \\
		50                & $SE_{a_{min}}(SV_i)$            & 56096.1                  & 59144.8                          \\
		60                & $SE_{a_{min}}(SV_i)$            & 114810.3                 & 114406.2                         \\
		70                & $SE_{a_{min}}(SV_i)$            & 169857.5                 & 168147.7                         \\
		80                & $SE_{a_{min}}(SV_i)$            & 270100.9                 & 274347.6                         \\
		90                & $SE_{a_{min}}(SV_i)$            & 378835.9                 & 372835.7                         \\
		100               & $SE_{a_{min}}(SV_i)$            & 507157.3                 & 510469                          
	\end{tabular}
\end{table}

\end{document}
