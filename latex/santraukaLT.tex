Paskutiniuose dviejuose dešimtmečiuose yra plačiai domimasi dinaminiais grafais, dėl jų naudos tokiose srityse kaip: komunikacijos tinkluose, VLSI kūrime, kompiuterinėje grafikoje. Šis darbas apima dinaminių grafų problemą, maksimalaus srauto radimą,viena iš labiausiai fundamentalių optimizavimo problemų. Šio darbo tikslas yra įrodyti, kad pateikiamo maksimalaus srauto radimo algoritmas yra efektyvesnis už statinio grafo Fordo Fulkersono algoritmą skaičiavimui panaudojamų briaunų atžvilgiu. Pateikiamo algoritmo veikimas yra pagrįstas modifikuotu grupavimo metodu ir modifikuotu Fordo Fulkersono algoritmu. Grupavimo metodas yra skirtas grafo išskirstymui į grupes, o modifikuotas Fordo Fulkersono algoritmas yra skirtas apskaičiuoti konkretaus regiono maksimaliems srautams. Atmintyje bus saugomas regionų išsidėstymas ir jų talpas grafo pavidale. Perskaičiavus grupes kuriuose įvyko pokytis, galima rasti pakitusio grafo maksimalų srautą. \\
Raktažodžiai: dinaminiai grafai, maksimalūs srautai,grupavimo metodas, Fordo Fulkersono algoritmas, euristiniai bandymai.