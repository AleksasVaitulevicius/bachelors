Šiame darbe tiriamas algoritmas yra paremtas Frederiksono suformuluotu grupavimo metodu \cite{DSfUoMST}. Grupavimo metodas - tai metodas, kuris yra pagrįstas grafo dalinimu į subgrafus vadinamus grupėmis. Grafas yra padalinamas taip, kad kiekviena atlikta operacija turėtų įtakos tik daliai grupių, bet ne visoms. Todėl tiriamas algoritmas veikia pagal grupavimo metodą tik tada kai yra patenkinta sąlyga: jei tinkle egzistuoja subgrafas, kuriame yra Eulerio ciklas, tai visos viršūnės priklausančios tam subgrafui yra vienoje grupėje. Šitai sąlygai patenkinti yra naudojama grupavimo funkcija, kuri naudoja šias pagalbines funkcijas:

Sąlygos tenkinimo funkcija - tinklo EG, kurio viršūnės yra grupių, tenkinančių pateiktą sąlygą, viršūnių masyvai, kūrimo funkcija.
\begin{enumerate}
	\item Iš apskaičiuojamo tinklo $C=\{V_C, E_C, u_C\}$ sukuriamas tinklas EG, kurio viršūnės būtų apskaičiuojamo tinklo viršūnės patalpintos masyvuose, o briaunos atitiktų apskaičiuojamo tinklo briaunas.
	\item Sukuriamas masyvas B, kuriame talpinamos briaunos, su kuriomis reikia daryti skaičiavimus, stekas PATH, kuriame talpinamos aplankytos viršūnės, ir viršūnių iteratorius x, jam suteikiama C šaltinio reikšmė.
	\item Jei PATH yra tuščias, einamana į žingsnį .
	\item Jei masyve B $\exists x \rightArrow y : y \in V_C$, tai sukuriamas masyvas B', į kurį yra sudedamos visos viršūnės y iš B masyvo.
	\item Jei masyve B $\nexists x \rightArrow y : y \in V_C$, tai sukuriamas masyvas B', į kurį yra sudedamos visos viršūnės y iš tinklo V.
	\item Jei B' yra tuščias tai einama į žingsnį .
	\item Iš masyvo B' yra išimamas pirmas elementas y.
	\item Jei $\nexists y \in V_C$, tai surandama viršūnė z, kuri turi visus y elementus (toliau y := z).
	\item Jei PATH neturi elemento y, tai elementas y įdedamas į PATH ir einamana į žingsnį .
	\item Inicializuojama nauja viršūnė n su visais y elementais.
	\item Iš PATH išimamas elementas z.
	\item Jei elementas z = y, tai enama į žingsnį 15 ir y = n.
	\item  Tinklo EG viršūnės z ir n yra pakeičiamos x ir n konkatenacija (toliau n yra z ir n konkatenacija).
	\item  Einama į žingsnį 11.
	\item  Visi masyvo B' elementai įdedami į masyvą B ir viršūnių iteratoriui x suteikiama y reikšmė.
	\item  Einama į žingsnį .
	\item  Iš steko PATH išimamas elementas, iteratoriui x suteikiama PATH viršutinio elemento reikšmė.
	\item  Einama į žingsnį .
\end{enumerate}

Pati grupavimo funkcija:
\begin{enumerate}
	\item Kviečiama Sąlygos tenkinimo funkcija.
\end{enumerate}
