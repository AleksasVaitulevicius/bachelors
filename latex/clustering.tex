Šiame darbe tiriamas algoritmas yra paremtas Frederiksono suformuluotu grupavimo metodu \cite{DSfUoMST}. Grupavimo metodas - tai metodas, kuris yra pagrįstas grafo dalinimu į subgrafus vadinamus grupėmis. Grafas yra padalinamas taip, kad kiekviena atlikta operacija turėtų įtakos tik daliai grupių, bet ne visoms. Todėl tiriamas algoritmas veikia pagal grupavimo metodą tik tada kai yra patenkinta sąlyga: jei tinkle egzistuoja subgrafas, kuriame yra Eulerio ciklas, tai visos viršūnės priklausančios tam subgrafui yra vienoje grupėje. Šitai sąlygai patenkinti yra naudojama grupavimo funkcija, kuri naudoja šias pagalbines funkcijas:

Subgrafų su Eulerio ciklais radimo funkcija - tinklo EG, kurio viršūnės yra grupių, tenkinančių pateiktą sąlygą (subgrafai su Eulerio ciklais), viršūnių masyvai, kūrimo funkcija.
\begin{enumerate}
	%1
	\item Iš apskaičiuojamo tinklo $C=\{V_C, E_C, u_C\}$ sukuriamas tinklas EG, kurio viršūnės būtų apskaičiuojamo tinklo viršūnės patalpintos masyvuose, o briaunos atitiktų apskaičiuojamo tinklo briaunas.
	%2
	\item Sukuriamas masyvas B, kuriame talpinamos briaunos, su kuriomis reikia daryti skaičiavimus, stekas PATH, kuriame talpinamos aplankytos viršūnės, ir viršūnių iteratorius x, jam suteikiama C šaltinio reikšmė ir patalpinamas į steką PATH.
	%3
	\item Jei PATH yra tuščias, einamana į žingsnį 19.
	%4
	\item Jei masyve B $\exists x \rightArrow y : y \in V_C$, tai sukuriamas masyvas B', į kurį yra sudedamos visos viršūnės y iš B masyvo.
	%5
	\item Jei masyve B $\nexists x \rightArrow y : y \in V_C$, tai sukuriamas masyvas B', į kurį yra sudedamos visos viršūnės y iš tinklo V.
	%6
	\item Jei B' yra tuščias tai einama į žingsnį 17.
	%7
	\item Iš masyvo B' yra išimamas pirmas elementas y.
	%8
	\item Jei $\nexists y \in V_C$, tai surandama viršūnė z, kuri turi visus y elementus (toliau y := z).
	%9
	\item Jei PATH neturi elemento y, tai elementas y įdedamas į PATH ir einamana į žingsnį 15.
	%10
	\item Inicializuojama nauja viršūnė n su visais y elementais.
	%11
	\item Iš PATH išimamas elementas z.
	%12
	\item Jei elementas z = y, tai enama į žingsnį 15 ir y = n.
	%13
	\item  Tinklo EG viršūnės z ir n yra pakeičiamos x ir n konkatenacija (toliau n yra z ir n konkatenacija).
	%14
	\item  Einama į žingsnį 11.
	%15
	\item  Visi masyvo B' elementai įdedami į masyvą B ir viršūnių iteratoriui x suteikiama y reikšmė.
	%16
	\item  Einama į žingsnį 3.
	%17
	\item  Iš steko PATH išimamas elementas, iteratoriui x suteikiama PATH viršutinio elemento reikšmė.
	%18
	\item  Einama į žingsnį 3.
	%19
	\item  Pabaiga.
\end{enumerate}

Srautui priklausančių grupių sukūrimo funkcija - iš pateikto tinklo $EG=\{V_EG, E_EG, u_EG\}$, kurio viršūnės yra apskaičiuojamo tinklo $C=\{V_C, E_C, u_C\}$ viršūnių, masyvai, sukuriamas grupių tinklas R.
\begin{enumerate}
	%1
	\item Sukuriamas stekas B, kuriame talpinamos viršūnės, kurios priklauso konkrečiai grupei, masyvas VR, kuriame talpinamos viršūnės, kurias reikia ištrinti iš tinklo EG, stekas NC, kuriame talpinamos viršūnės kitų grupių kūrimui, ir viršūnių iteratorius x, jam suteikiama EG tinklo viršūnės, kurioje yra tinklo C šaltinis, reikšmė ir patalpinamas į steką NC.
	%2
	\item Inicializuojama nauja grupė G.
	%3
	\item Jei NC yra tuščias, einamana į žingsnį 21.
	%4
	\item Viršutinis NC elementas yra perkeliamas į B ir iteratoriui x yra suteikiama to elemento reikšmė.
	%5
	\item Jei x dydis = 1, tai į tinklą G patalpinama elemento x reikšmė ir einama į žingsnį 10.
	%6
	\item Tinklui G priskiriama reikšmė $G=\{x, Y, Y_u\}$, kur Y yra tinklo C briaunos tarp x viršūnių, o $Y_u$ - tai tų briaunų svoriai. 
	%7
	\item Naudojantis apibrėžimu sukuriamos tinklo R briaunos $G \rightarrow Q$, kur Q yra menamos viršūnės, nustatomi grupės G šaltiniai ir tikslai.
	%8
	\item Tinklas G patalpinamas į tinklą R.
	%9
	\item Tinklui G inicializuojama naujos grupės reikšmė.
	%10
	\item Jei B yra tuščias einama į žingsnį 16.
	%11
	\item Iš steko B išimamas viršutinis elementas, kurio reikšmė priskiriama iteratoriui x.
	%12
	\item Jei $\exists y : x \rightarrow y$, y dydis > 1, $y \notin NC$ ir y nėra NC viršūnė, tai y talpinamas į steką NC ir einama į žingsnį 12.
	%13
	\item Jei $\exists y : x \rightarrow y$, y dydis = 1, $y \notin NC$ ir y nėra NC viršūnė, tai y elemento reikšmė patalpinama į steką B, to elemento reikšmė pridedama į grupę G kaip viršūnė ir einama į žingsnį 12.
	%14
	\item Į steką VR įdedama iteratoriaus x reikšmė.
	%15 
	\item Einama į žingsnį 10.
	%16
	\item Tinklui G priskiriamos briaunos tinklo C briaunos, kurių viršūnės atitinka G viršūnes. 
	%17
	\item Naudojantis apibrėžimu sukuriamos tinklo R briaunos $G \rightarrow Q$, kur Q yra menamos viršūnės, nustatomi grupės G šaltiniai ir tikslai.
	%18
	\item Tinklas G patalpinamas į tinklą R.
	%19
	\item Tinklui G inicializuojama naujos grupės reikšmė.
	%20
	\item  Einama į žingsnį 3.
	%21
	\item  Iš tinklo EG ištrinamos visos viršūnės, kurios yra VR masyve.
	%22
	\item  Pabaiga.
\end{enumerate}

Pati grupavimo funkcija:
\begin{enumerate}
	\item Kviečiama subgrafų su Eulerio ciklais radimo funkcija (rezultatas tinklas EG).
	\item Kviečiama srautui priklausančių grupių sukūrimo funkcija su tinklu EG.
	\item Likusios tinklo EG viršūnės konkatenuojamos į masyvą X.
	\item Sukuriamas tinklas $G=\{X, Y, Y_u\}$, kur Y yra tinklo C briaunos tarp X viršūnių, o $Y_u$ - tai tų briaunų svoriai. 
	\item Naudojantis apibrėžimu sukuriamos tinklo R briaunos $G \rightarrow Q$, kur Q yra menamos viršūnės, nustatomi grupės G šaltiniai ir tikslai.
	\item Tinklas G patalpinamas į tinklą R.
	\item Pabaiga.
\end{enumerate}
