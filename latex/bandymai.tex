Tam, kad nustatyti ar verta formaliai įrodyti sukurtą algoritmą buvo atlikti empiriniai bandymai, kurie skirti ištirti algoritmo korektišką veikimą ir efektyvumą. Šiems bandymams atlikti buvo įgyvendintas sukurtas algoritmas, modulis, generuojanti tinklus, ir modulis, kuris atlieka bandymus. 

Modulis, generuojantis tinklus, buvo panaudotas sugeneruoti tinklų $N_i = \{V_{N_i}, E_{N_i}, u{N_i}\}$ aibę A. Aibėje A yra 10 poaibių $A'_j$, kurių tinklų viršūnių aibių $V_{N_i}$ dydžiai $SV_i$ yra lygūs. Tad kiekvienas poaibis  $A'_j$ turi skirtingą dydį  $SV_j$. Modulis sugeneruoja poaibius $A'_j$ su šiais $SV_j = 10 \times j : j = 1 .. 10$. Kiekviename poaibyje $A'_j$ yra po 3 poaibius $A''_j$, kuriuose yra po 10 tinklų $N_i$, kurių briaunų aibių $E_{N_i}$ dydžiai yra lygūs. Tad kiekvienas poaibis  $A''_j$ turi skirtingą dydį  $SE_j$. Modulis sugeneruoja poaibius $A''_j$ su šiais  $SE_j : SE_1 = \frac{SV_i^2 + 2 \times SV_i - 3}{4}, SE_2 = \frac{SV_i^2 - 1}{2}, SE_3 = \frac{3 \times SV_i^2 - 2 \times SV_i - 1}{4}$