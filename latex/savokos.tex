\begin{enumerate}
	\item Dinaminiai grafai (angl. Dynamic graphs) - tai grafai, kuriuose galima atlikti tokias operacijas: pridėjimo (pridėti briauną, pridėti viršūnę), atėmimo (atimti briauną. atimti viršūnę), papildomos operacijos priklausomai nuo grafo savybių (pavyzdžiui jei grafas yra svorinis, tai jis galėtų turėti papildomą operaciją keisti svorį).
	\item Dalinai dinaminis grafas (angl. Partial dynamic graph) - tai dinaminis grafas, kuriame yra uždrausta bent viena, bet ne visos dinaminio grafo operacijos.
	\item Pilnai dinaminis grafas (angl. Fully dynamic graph) - tai dinaminis grafas, kuriame yra leidžiamos visos operacijos.
	\item Inkrementalus dinaminis grafas (angl. Incremental dynamic graph) - dinaminis grafas, kuriame yra leidžiama pridėjimo operacijos.
	\item Dekrementalus dinaminis grafas (angl. Decremental dynamic graph) - dinaminis grafas, kuriame leidžiamos atėmimo operacijos.
	\item Grupė (angl. Cluster) - tinklo subgrafas, kuris yra naudojamas grupavimo metode.
	\item Grupavimo metodas (angl. Clustering) - algoritmo dinaminiui grafui konstravimo būdas, kuriame grafas yra suskirstomas į grupes.
	\item Tinklas (angl. Network) - orientuotas grafas, kurio briaunos turi talpas.
	\item Srautas (angle. Flow) - tai leistinas kelias konkrečiam kiekiui iš šaltinio į tikslą.
	\item Šaltinis (angl. Source) - tinklo viršūnė, kuri yra srauto pradžios taškas.
	\item Tikslas (angl. Sink) - tinklo viršūnė, kuri yra srauto pabaigos taškas.
	\item Talpa (angl. Capacity) - briaunos savybė, kuri nurodo koks kiekis gali ja praeiti.
	\item  Maksimalus srautas (angl. Max flow) - srautas, kurio dydis yra didžiausias iš visų leistinų srautų.
	\item Vieno šaltinio ir kelių tikslų maksimalus srautas (angl. Single source multi sink max flow) - tai maksimalaus srauto tipas, kuriame yra vienas šaltinis ir daugiau nei vienas tikslas.
	\item Kelių šaltinių ir vieno tikslo maksimalus srautas (angl. Multi source single sink max flow) - tai maksimalaus srauto tipas, kuriame yra daugiau nei vienas šaltinis ir vienas tikslas.
	\item Kelių šaltinių ir kelių tikslų maksimalus srautas (angl. Multi source multi sink max flow) - tai maksimalaus srauto tipas, kuriame yra daugiau nei vienas šaltinis ir daugiau nei vienas tikslas.
	\item Vieno šaltinio ir vieno tikslo maksimalus srautas (angl. Single source single sink max flow) - tai maksimalaus srauto tipas, kuriame yra vienas šaltinis ir vienas tikslas.
	\item BFS (angl. Breadth first search) - paieška į plotį.
	%Below ones have to be defined
	\item Euristinis eksperimentas (angl. heuristic experiment) -
\end{enumerate}