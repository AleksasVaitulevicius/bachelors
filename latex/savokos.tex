\begin{enumerate}
	\item Dinaminiai grafai (angl. Dynamic graphs) - tai grafai, kuriuose galima atlikti tokias operacijas:
	\begin{enumerate}
		\item Pridėti briauną
		\item Atimti briauną
		\item Pridėti viršūnę
		\item Atimti viršūnę
		\item Papildomos operacijos (pakeisti briaunos spalvą, pralaidumą ir panašiai).
	\end{enumerate}
	\item Dalinai dinaminis grafas (angl. Partial dynamic graph) - tai dinaminis grafas, kuriame yra uždrausta bent viena, bet ne visos dinaminio grafo operacijos.
	\item Pilnai dinaminis grafas (angl. Fully dynamic graph) - tai dinaminis grafas, kuriame yra leidžiamos visos operacijos.
	\item Inkrementalus dinaminis grafas (angl. Incremental dynamical graph) - dinaminis grafas, kuriame yra leidžiama pridėjimo operacijos.
	\item Dekrementalus dinamainis grafas (angl. Decremental dynamical graph) - dinaminis grafas, kuriame leidžiamos atėmimo operacijos.
	\item Grupė (angl. Cluster) - tinklo pografis, kuris yra naudojamas grupavimo metode.
	\item Grupavimo metodas (angl. Clustering) - algoritmo dinaminiui grafui konstravimo būdas, kuriame grafas yra suskirstomas į grupes.
	\item Tinklas (angl. Network) - orientuotas grafas, kurio briaunos turi pralaidumus.
	\item Srautas (angle. Flow) - tai leistinas kelias konkrečiam kiekiui iš šaltinio į tikslą.
	\item Šaltinis (angl. Source) - tinklo viršūnė, kuri yra srauto pradžios taškas.
	\item Tikslas (angl. Sink) - tinklo viršūnė, kuri yra srauto pabaigos taškas.
	\item Pralaidumas (angl. Capacity) - briaunos savybė, kuri nurodo koks kiekis gali ja praeiti.
	\item  Maksimalus srautas (angl. Max flow) - srautas, kurio dydis yra didžiausias iš visų leistinų srautų.
	\item Vienšaltinis daugtikslinis maksimalus srautas (angl. Single source multi sink max flow) - tai maksimalaus srauto tipas, kuriame yra vienas šaltinis ir daugiau nei vienas tikslas.
	\item Daugiašaltinis vientikslinis maksimalus srautas (angl. Multi source single sink max flow) - tai maksimalaus srauto tipas, kuriame yra vienas šaltinis ir vienas tikslas.
	\item Daugiašaltinis daugtikslinis maksimalus srautas (angl. Multi source multi sink max flow) - tai maksimalaus srauto tipas, kuriame yra daugiau nei vienas šaltinis ir daugiau nei vienas tikslas.
	\item Vienšaltinis vientikslinis maksimalus srautas (angl. Single source single sink max flow) - tai maksimalaus srauto tipas, kuriame yra daugiau nei vienas šaltinis ir daugiau nei vienas tikslas.
\end{enumerate}