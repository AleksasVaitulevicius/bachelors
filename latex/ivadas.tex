Paskutiniuose dviejuose dešimtmečiuose yra plačiai domimasi dinaminiais grafais. Ši sritis suteikia daug teorinių žinių, kurios gali būti pritaikytos optimizuojant tokias sritis kaip: komunikacijos tinklus, VLSI kūrimą, kompiuterinę grafiką, kaip yra teigiama publikacijose \cite{DynamicGraphs, DGA}. Šiame darbe bus tiriamas algoritmas skirtas maksimaliam srautui rasti dinaminiuose tinkluose.

Dinaminis grafas - tai grafas, kuriam yra galima atlikti bent vieną iš šių operacijų:
\begin{itemize}
	\item Pridėjimo
	\begin{itemize}
		\item Pridėti briauną.
		\item Pridėti viršūnę.
	\end{itemize}
	\item Atėmimo
	\begin{itemize}
		\item Atimti briauną.
		\item Atimti viršūnę.
	\end{itemize}
	\item  Papildomos operacijos priklausomai nuo grafo savybių(pavyzdžiui jei grafas yra svorinis, tai jis galėtų turėti papildomą operaciją keisti svorį).
\end{itemize}
Pagal leidžiamas operacijas dinaminiai grafai yra skirstomi į:
\begin{itemize}
	\item dalinai dinaminius grafus, kurie yra skirstomi į:
	\begin{itemize}
		\item inkrementalus - vykdoma tik pridėjimo operacija
		\item dekrementalus - Vykdoma tik atėmimas
	\end{itemize}
	\item  pilnai dinaminius grafus, kuriuose vykdomos visos operacijos.
\end{itemize}
Šiame darbe tiriamas algoritmas yra skirtas spręsti pilnai dinaminio grafo uždavinį.

Visa statinio grafo problemų aibė yra dinaminio grafo problemų poaibis. Tačiau dinaminiame grafo problemos sprendimas gali būti optimizuotas, nes yra daugiau informacijos apie grafą nei statiniame grafe (pavyzdžiui, jei buvo apskaičiuotas grafo maksimalus srautas ir prie grafo buvo pridėta briauna, tai bus žinomas grafo poaibio, kuriam nepriklauso naujai pridėta briauna, maksimalus srautas). Šiame darbe tiriamas algoritmas sprendžia maksimalaus srauto problemą, kuri priklauso šiai aibei.

Maksimalus srautas - tai didžiausias galimas srautas tinkle iš viršūnių $s_i$ (šaltinių) iki viršūnių $t_i$ (tikslų). Tinklas - tai orientuotas grafas $G= \{V, E, u\}$, kur V yra viršūnių aibė, E - briaunų aibė, o u - briaunų pralaidumų aibė $( u : E \rightarrow R )$. Dinaminio tinklo apibrėžimas yra analogiškas dinaminio grafo apibrėžimui. Pagal šaltinių ir tikslų skaičių ši problema yra skirstoma į:
\begin{itemize}
	\item Vienšaltinę daugtikslinę - tai srautas, kuriame yra vienas šaltinis ir daugiau nei vienas tikslas.
	\item Daugiašaltinę vientikslinę - tai srautas, kuriame yra vienas tikslas ir daugiau nei vienas šaltinis.
	\item Daugiašaltinę daugtikslinę - tai srautas, kuriame yra daugiau nei vienas šaltinis ir daugiau nei vienas tikslas.
	\item Vienšaltinę vientikslinę - tai srautas, kuriame yra vienas šaltinis ir vienas tikslas.
\end{itemize}
Algoritmas, tiriamas šiame darbe, sprendžia tik vienšaltinę vientikslinę problemą. Tačiau tarpinėms reikšmėms gauti yra naudojama ir likusių problemų sprendimo būdas, Fordo Fulkersono algoritmas pritaikytas spręsti daugiašaltinę daugtikslinę problemą. Tiriamo algoritmo veikimo principas yra pagrįstas grupavimo metodu. Tai reiškia, kad tinklas yra padalinamas į grupes. Šių grupių maksimaliams srautams rasti ir yra naudojamas modifikuotas Fordo Fulkersono algoritmas.

Šiame darbe tiriamo algoritmo korektiškumas nėra įrodytas. Jo veikimo korektiškumą gali pagrįsti tik matematinis įrodymas. Tačiau matematinis įrodymas yra sudėtingas procesas. Tad reikia atlikti empirinius tyrimus su algoritmu, kurie parodytų ar verta tirti algoritmą. Taip pat nėra nustatyta ar tiriamas algoritmas yra efektyvesnis už algoritmą randantį maksimalų srautą statiniame tinkle. Jei tiriamas algoritmas nėra efektyvesnis, tai jo įrodyta neapsimoka, nes algoritmai skirti statiniams grafams, gali būti pritaikyti ir dinaminiams. Tad algoritmai, skirti dinaminiams grafams, yra naudojami tik tuo atveju jei jie yra efektyvesni už algoritmus, skirtus statiniams grafams.

\newpage

Tad šio darbo \textbf{TIKSLAS} yra įgyvendinti empirinį tyrimą, kuris įrodytų, kad šiame darbe tiriamą algoritmą yra verta matematiškai ištirti. Šiam tikslui pasiekti reikia atlikti šiuos uždavinius:
\begin{enumerate}
	\item Pateikti tiriamą algoritmą:	
	\begin{enumerate}
		\item Pateikti grupės, į kurias bus sugrupuotas grafas, apibrėžimą.
		\item  Pateikti Fordo Fulkersono algoritmą ir kaip jis buvo pritaikytas daugšaltinei daugtikslinei problemai.
		\item Pateikti grupavimo funkciją.
		\item Apibrėžti funkciją, kuri randa viso tinklo maksimalų srautą su pakitusiu bent vienu maksimaliu srautu vienoje ar keliose grupėse. Jei buvo įvykdyta operacija:
		\begin{enumerate}
			\item Pridėti viršūnę
			\item Pridėti briauną
			\item Atimti viršūnę
			\item Atimti briauną
			\item Pakeisti briaunos pralaidumą
		\end{enumerate}
		\item Pateikti tiriamo algoritmo pagrindinę funkciją.
	\end{enumerate}
	\item Pateikti tiriamo algoritmo panaudojimo pavyzdį.
	\item Įgyvendinti tiriamą algoritmą.
	\item Įgyvendinti Fordo Fulkersono algoritmą, skirtą rasti maksimalų srautą statiniame tinkle.
	\item Atlikti empirinius bandymus:
		\begin{enumerate}
			\item Surinkti šiuos duomenis apie tiriamą algoritmą ir įgyvendintą algoritmą, skirtą rasti maksimalų srautą statiniame tinkle:
			\begin{enumerate}
				\item Abiejų algoritmų skaičiavimų rezultatus.
				\item Abiejų algoritmų skaičiavimuose panaudotų briaunų kiekius.
				\item Abiejų algoritmų skaičiavimuose panaudotų viršūnių  kiekius.
			\end{enumerate}
			\item Ištirti koreliaciją tarp briaunų panaudotų skaičiavimuose tiriamo algoritmo ir tinklo, naudojamo skaičiavimuose, briaunų kiekius. 
			\item Ištirti koreliaciją tarp briaunų panaudotų skaičiavimuose tiriamo algoritmo ir tinklo, naudojamo skaičiavimuose, viršūnių kiekius. 
			\item Ištirti koreliaciją tarp briaunų panaudotų skaičiavimuose Fordo Fulkersono algoritmo ir tinklo, naudojamo skaičiavimuose, briaunų kiekius. 
			\item Ištirti koreliaciją tarp briaunų panaudotų skaičiavimuose Fordo Fulkersono algoritmo ir tinklo, naudojamo skaičiavimuose, viršūnių kiekius. 
			\item Palyginti tiriamo algoritmo ir Fordo Fulkersono algoritmo rezultatus.
			\item Palyginti tiriamo algoritmo ir Fordo Fulkersono algoritmo koreliacijas.
	\end{enumerate}
\end{enumerate}

Darbas susideda iš dviejų dėstymo skyrių. Pirmame skyriuje yra pateikiamas tiriamas algoritmas. Pirmo skyriaus pirmame poskyryje yra pateikiamas grupės apibrėžimas, tam, kad žinoti į kokius subgrafus bus skaidomi tinklai tiriamame algoritme. Antrame poskyryje yra pateikiamas Fordo Fulkersono algoritmas ir kaip jis buvo pritaikytas maksimaliems srautams rasti grupėse. Trečiame poskyryje pateikiama grupavimo funkcija. Ketvirtas poskyris yra skirtas pateikti visoms funkcijoms, kurios būna iškviečiamos  atlikus pridėjimo, atėmimo arba briaunos pralaidumo pakeitimo operacijas. Paskutiniame poskyryje pirmo skyriaus yra apibrėžiama funkcija, kuri apskaičiuoja visą informaciją, kuri yra  reikalinga funkcijoms, kurios yra iškviečiamos atlikus pridėjimo, atėmimo arba briaunos pralaidumo pakeitimo operacijas. Antrame skyriuje yra aprašomi empiriniai bandymai ir pateikiami jų rezultatai.