Paskutiniuose dviejuose dešimtmečiuose yra plačiai domimasi algoritmais skirtais spręsti įvairias problemas dinaminiuose grafuose. Šių problemų sprendimai optimizuoja tokias sritis kaip: komunikacijos tinklus, VLSI kūrimą, kompiuterinę grafiką \cite{DynamicGraphs, DGA} Šios problemos yra statinių grafų problemų poaibis. Tačiau dinaminių grafų problemų sprendimai gali būti labiau optimizuoti nei statinių, nes yra daugiau informacijos apie grafą nei statiniame grafe (pavyzdžiui, jei buvo apskaičiuotas grafo maksimalus srautas ir prie grafo buvo pridėta briauna, tai bus žinomas grafo poaibio, kuriam nepriklauso naujai pridėta briauna, maksimalus srautas). Vienai iš šių problemų, maksimalaus srauto paieškos problemai, buvo sukurtas algoritmas kursiniame darbe. Sukurtasis algoritmas šiame darbe buvo  įgyvendintas ir ištirtas. Įgyvendinimo metu buvo pastebėta daug netikslumų, tad algoritmas buvo pataisytas.

Dinaminis grafas - tai grafas, kuriam yra galima atlikti bent vieną iš šių operacijų: pridėjimo (pridėti briauną, pridėti viršūnę), atėmimo (atimti briauną, atimti viršūnę), papildomos operacijos priklausomai nuo grafo savybių (pavyzdžiui jei grafas yra svorinis, tai jis galėtų turėti papildomą operaciją keisti svorį). Pagal leidžiamas operacijas dinaminiai grafai yra skirstomi į: dalinai dinaminius grafus, kurie yra skirstomi į inkrementalius (vykdoma tik pridėjimo operacija) ir dekrementalius (vykdoma tik atėmimas), ir pilnai dinaminius grafus, kuriuose vykdomos visos operacijos. Sukurtas algoritmas yra skirtas spręsti pilnai dinaminio grafo uždavinį.

Maksimalus srautas - tai didžiausias galimas srautas tinkle iš viršūnių $s_i$ (šaltinių) iki viršūnių $t_i$ (tikslų). Tinklas - tai orientuotas grafas $G= \{V, E, u\}$, kur V yra viršūnių aibė, E - briaunų aibė, o u - briaunų talpų aibė $( u : E \rightarrow R )$. Pagal šaltinių ir tikslų skaičių ši problema yra skirstoma į:
\begin{itemize}
	\item Vieno šaltinio ir kelių tikslų - tai srautas, kuriame yra vienas šaltinis ir daugiau nei vienas tikslas.
	\item Kelių šaltinių ir vieno tikslo - tai srautas, kuriame yra vienas tikslas ir daugiau nei vienas šaltinis.
	\item Kelių šaltinių ir kelių tikslų  - tai srautas, kuriame yra daugiau nei vienas šaltinis ir daugiau nei vienas tikslas.
	\item Vieno šaltinio ir vieno tikslo - tai srautas, kuriame yra vienas šaltinis ir vienas tikslas.
\end{itemize}

Sukurtas algoritmas gali rasti  tik vieno šaltinio ir kelių tikslų maksimalų srautą. Tačiau tarpinėms reikšmėms gauti algoritmas turi rasti ir visų kitų tipų maksimalius srautus. Tam naudojamas Fordo Fulkersono algoritmas  \cite{FiN} pritaikytas rasti kelių šaltinių ir kelių tikslų maksimalius srautus. Taip modifikuotas Fordo Fulkersono algoritmas gali rasti maksimalius srautus grupėse. Grupės - tai tinklo, kuriame ieškomas maksimalus srautas, subgrafai. Metodas, kuriame yra naudojamos grupės yra vadinamas grupavimo metodu \cite{DSfUoMST}, kuriame tinklas yra padalinamas į grupes. Šis metodas buvo pritaikytas konstruojant algoritmą.

Sukurto algoritmo korektiškumui įrodyti reikia pateikti formalų įrodymą. Tačiau formalus įrodymas yra sudėtingas procesas. Todėl verta ištirti algoritmo korektišką veikimą naudojantis empiriniais tyrimais prieš formalų įrodymą. Šie tyrimai suteikia galimybę lengvai nustatyti ar sukurtas algoritmas veikia nekorektiškai. Jei empiriniais tyrimais yra nustatomas nekorektiškas algoritmo veikimas, tai reiškia, kad formalus įrodymas yra neprasmingas. Tad šiame darbe yra atliekami empiriniai tyrimai, o ne formalus įrodymas. Taip pat nėra nustatyta ar sukurtas algoritmas yra efektyvesnis už algoritmą randantį maksimalų srautą statiniame tinkle. Jei sukurtas algoritmas nėra efektyvesnis, tai jo formalus įrodymas yra neprasmingas, nes algoritmai skirti statiniams grafams, gali būti pritaikyti ir dinaminiams. Tad algoritmai, skirti dinaminiams grafams, yra naudojami tik tuo atveju jei jie yra efektyvesni už algoritmus, skirtus statiniams grafams.

Tad šio darbo \textbf{TIKSLAS} yra nustatyti ar sukurtą algoritmą verta formaliai įrodyti. Šiam tikslui pasiekti reikia atlikti šiuos uždavinius:
\begin{enumerate}
	\item Pateikti sukurtą algoritmą:	
	\begin{enumerate}
		\item  Pateikti Fordo Fulkersono algoritmą ir kaip jis buvo pritaikytas kelių šaltinių ir kelių tikslų problemai.
		\item Pateikti grupavimo funkciją.
		\item Pateikti funkciją, kuri randa viso tinklo maksimalų srautą su pakitusiu bent vienu maksimaliu srautu vienoje ar keliose grupėse.
		\item Pateikti sukurto algoritmo pagrindines funkcijas.
	\end{enumerate}
	\item Pateikti sukurto algoritmo panaudojimo pavyzdžius.
	\item Įgyvendinti sukurtą algoritmą.
	\item Įgyvendinti ir atlikti empirinius tyrimus.
	\item Atlikti statistinius skaičiavimus su atliktų tyrimų rezultatais.
\end{enumerate}

Darbas susideda iš trijų dėstymo skyrių. Pirmame skyriuje yra pateikiamas sukurtas algoritmas (1. užduotis). Jame išdėstoma kaip Fordo Fulkersono algoritmas buvo pritaikytas sukurtam algoritmui (1.a užduotis), grupavimo funkcija (1.b užduotis), funkcija, kuri randa viso tinklo maksimalų srautą su pakitusiu bent vienu maksimaliu srautu vienoje ar keliose grupėse (1.c užduotis), ir sukurto algoritmo pagrindinė funkcija (1.d užduotis). Antrame skyriuje pateikiami sukurto algoritmo pavyzdžiai (2 užduotis). Trečiame skyriuje yra aprašomi empiriniai bandymai, pateikiami jų rezultatai (4 užduotis), aprašomi atlikti statistiniai skaičiavimai ir jų rezultatai (5 užduotis).